\chapter{Background Survey}\label{C:backgroundsurvey}
\section{Core Concepts}
\subsection{CNF \& DNF}
A boolean formula is in Conjunctive Normal Form (CNF) if and only if it is a conjunction (and) of clauses. A clause in a CNF formula is given by a disjunction (or ) of literals. A literal is either an atom or the negation of an atom, an atom is one of the variables in the formula.\\

Consider the boolean formula $\lnot a \lor (b \land c)$, the CNF is $(\lnot a \lor b) \land (\lnot a \lor c)$. In this CNF formula the clauses are $(\lnot a \lor b)$, $(\lnot a \lor c)$, the literals used are $\lnot a$, $b$, $c$ and the atoms are $a$, $b$, $c$.\\

A boolean formula is in Disjunctive Normal Form (DNF) if and only if it is a disjunction (or) of clauses. A DNF clause is a conjunction (and) of literals. Literals and atoms are defined the same as in CNF formulas.\\

Consider the boolean formula $\lnot a \land (b \lor c)$, the DNF is $(\lnot a \land b) \lor (\lnot a \land c)$.\\

\subsection{CNF \& DNF from Truth Table}
Given a truth table representing a boolean formula, constructing a DNF formula involves taking all rows which correspond to True and combining them with an OR operation. To construct a CNF one combines the negation of any row which corresponds to False by an OR operation and negates it.

\section{Literature Review}

A survey in 1995 focuses on rule extraction algorithms \cite{andrews1995survey}, identifying the reasons for needing these algorithms along with introducing ways to categorise and compare them. Motivation behind scientific study is always crucial so why is understanding the knowledge contained inside Artificial Neural Networks's (ANN's) important? The key points identified are that the ANN might of discovered some rule or patten in the data which is currently not known, being able to extract these rules would give humans a greater understanding of the problem. Another, perhaps more significant reason is the application of ANN's to systems which can effect the safety of human lives, i.e. Aeroplanes, Cars. If using an ANN in the context of a system involving human safety it is important to be certain of the knowledge inside the network, to be sure that the ANN wont take any dangerous actions.\\

There are three categories that rule extraction algorithms fall into \cite{andrews1995survey}. An algorithm in the \textbf{decompositional} category focuses on extracting rules from each hidden/output unit. If an algorithm is in the \textbf{pedagogical} category then rule extraction is thought of as a learning process, the ANN is treated as a black box and the algorithm learns a relationship between the input and output vectors. The third category, \textbf{electic}, is a combination of decompositional and pedagogical. Electic accounts for algorithms which inspect the hidden/output neurons individually but extracts rules which represent the ANN globally \cite{tickle1998truth}.\\

To further divide the categories two more classifications are introduced. One measures the portability of rule extraction techniques, i.e. how easily can they be applied to different types of ANN's. The second is criteria to assess the quality of the extracted rules, these are accuracy, fidelity, consistency, comprehensibility \cite{andrews1995survey}.

\begin{enumerate}
\item A rule set is \textbf{Accurate} if it can generalize, i.e. classify previously unseen examples.
\item The behaviour of a rule set with a high \textbf{fedelity} is close to that of the ANN it was extracted from.
\item A rule set is \textbf{consistent} if when trained under different conditions it generates rules which assign the same classifications to unseen examples.
\item The measure of \textbf{comprehensibility} is defined by the number of rules in the set and the number of literals per rule.
\end{enumerate}

The paper "Backpropagation for Neural DNF- and CNF-Networks" presents an approach which relies on a special NN architecture. The neurons in these networks have a restricted function space, they are only able to perform a OR or AND on a subset of their inputs. By restricting the degrees of freedom in the network it is possible to understand the actions each neuron is taking. Rules can simply be extracted from the trained network by inspecting these neurons \cite{herrmann1996backpropagation}. The conjunctive and disjunctive neurons presented, while making sense mathematically are cumbersome to implement. A perfected way to describe logical neurons will be to use Noisy-OR and Noisy-AND gates \cite{LearningLogicalActivations}, derived from the Noisy-OR relation which was developed by Judea Pearl \cite{russell1995modern}, a concept in Bayesian Networks. \\

A Bayesian Network represents the conditional dependencies between random variables in the form of a directed acyclic graph.

\begin{figure}[H]
  \centering
  \begin{minipage}[b]{0.4\textwidth}
    \includegraphics[width=\textwidth]{bayesian-network-example.png}
    \caption{}
    \label{fig:bayesian-network-example}
  \end{minipage}
  \hfill
\end{figure}

Figure \ref{fig:bayesian-network-example} is a Bayesian network, it demonstrates the dependency between random variables "Rush Hour", "Raining", "Traffic", "Late To Work". The connections show dependencies i.e. Traffic influences whether you are late to work, and it being rush hour or raining influences whether there is traffic.\\

Consider a Bayesian network having the following configuration, take some node $D$ with $S_1,..., S_n$ as parents i.e. $S_i$ influences the node $D$, each $S_i$ is independent from all others. The relationship between D and its parents is if $S_1\ OR\ ...\ OR\ S_n$ is true then $D$ is true. Let $\epsilon_i$ be the uncertainty that $S_i$ influence $D$ then $P(D = 1| S_1 = 1, , S_n = 1)$ can be defined.

\begin{align}
P(D = 1 | S_1 = 1, ..., S_n = 1) = 1 - \prod^n_{i=1} \epsilon_i
\end{align}

In the context of a neuron, the inputs $x_1, ..., x_n$ represent the probability that inputs $1, ..., n$ are true. Each $\epsilon_i$ is the uncertainty as to whether $x_i$ influences the output of the neuron. How can weights and inputs be combined to create a final activation value for the neuron. First consider a function $f(\epsilon, x)$ which computes the irrelevance of input x. Some conditions that can be placed on $f$ are given in \cite{LearningLogicalActivations}. (1) $\epsilon = 1$ means that $f(\epsilon, x) = 1$, (2) $x = 1$ means that $f(\epsilon, x) = 1$, (3) Monotonically increasing in $\epsilon$ and decreasing in x. Let $f(x, \epsilon) = \epsilon^x$. The definitions for Noisy-OR and Noisy-AND gates can now be given.

\begin{definition}
A \textbf{Noisy-OR} Neuron has weights $\epsilon_1, ..., \epsilon_n \in [0,1]$ which represent the irrelevance of corresponding inputs $x_1, ..., x_n \in (0,1]$. The activation of a Noisy-OR Neurons is.

\begin{align}
a = 1 - \prod^p_{i=1} (\epsilon_i^{x_i}) \cdot \epsilon_b
\label{equ:noisy-or-activation-1}
\end{align}
\end{definition}

\begin{definition}
A \textbf{Noisy-AND} Neuron has weights $\epsilon_1, ..., \epsilon_n \in [0, 1]$ which represent the irielevence of corresponding inputs $x_1, ..., x_n \in (0,1]$. The activation of a Noisy-AND Neurons is.

\begin{align}
a = \prod^p_{i=1} (\epsilon_i^{1 - x_i}) \cdot \epsilon_b
\label{equ:noisy-and-activation-1}
\end{align}
\end{definition}

Both these parametrisations reduce to discrete logic gates when there is no noise, i.e. $\epsilon_i = 0$ for all $i$.\\

While the concept presented in \cite{herrmann1996backpropagation} is the foundation for the work presented in this report, the approach presented is different that what has been done. A different approach to disjunctive and conjunctive neurons is taken, along with this more investigation is carried out in terms of the capabilities of these networks (when compared to a standard perceptron) and the rule extraction method.