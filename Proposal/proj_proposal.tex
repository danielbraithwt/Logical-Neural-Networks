%% $RCSfile: proj_proposal.tex,v $
%% $Revision: 1.3 $
%% $Date: 2016/06/10 03:44:08 $
%% $Author: kevin $

\documentclass[11pt, a4paper, twoside, openright]{report}

\usepackage{float} % lets you have non-floating floats

\usepackage{url} % for typesetting urls

%  We don't want figures to float so we define
%
\newfloat{fig}{thp}{lof}[chapter]
\floatname{fig}{Figure}

%% These are standard LaTeX definitions for the document
%%
\title{Logical Neural Networks: Opening the black box}
\author{Daniel Thomas Braithwaite}

%% This file can be used for creating a wide range of reports
%%  across various Schools
%%
%% Set up some things, mostly for the front page, for your specific document
%
% Current options are:
% [ecs|msor|sms]          Which school you are in.
%                         (msor option retained for reproducing old data)
% [bschonscomp|mcompsci]  Which degree you are doing
%                          You can also specify any other degree by name
%                          (see below)
% [font|image]            Use a font or an image for the VUW logo
%                          The font option will only work on ECS systems
%
\usepackage[image,ecs,bschonscomp]{vuwproject} 

% You should specifiy your supervisor here with
%     \supervisor{Firstname Lastname}
% use \supervisors if there are more than one supervisor
\supervisor{Marcus Frean}

% Unless you've used the bschonscomp or mcompsci
%  options above use
%   \otherdegree{OTHER DEGREE OR DIPLOMA NAME}
% here to specify degree

% Comment this out if you want the date printed.
\date{}

\begin{document}

% Make the page numbering roman, until after the contents, etc.
\frontmatter

%%%%%%%%%%%%%%%%%%%%%%%%%%%%%%%%%%%%%%%%%%%%%%%%%%%%%%%

\begin{abstract}
  This document gives some ideas about how to write a project
  proposal, and provides a template for a proposal. You should discuss
  your proposal with your supervisor.
\end{abstract}

%%%%%%%%%%%%%%%%%%%%%%%%%%%%%%%%%%%%%%%%%%%%%%%%%%%%%%%

\maketitle

%\tableofcontents

% we want a list of the figures we defined
%\listof{fig}{Figures}

%%%%%%%%%%%%%%%%%%%%%%%%%%%%%%%%%%%%%%%%%%%%%%%%%%%%%%%

\mainmatter

%%%%%%%%%%%%%%%%%%%%%%%%%%%%%%%%%%%%%%%%%%%%%%%%%%%%%%%

\section*{1. Introduction}

Neural Networks peform exceptonallty well over a wide range of different problems. However a problelm is that once trained these networks become a black box, near impossible for a human to understand what features the network is using to solve the problem presented to it. Logical Neural Netowrks (neural networks with logical activation functions) have been shown to provide a more understandable representation.

\section*{2. The Problem}

Continueing on from previous work giving a proof of concept for LLN's, showing (in the case of the MINST dataset) that a trained LLN has significant improvements in the ease at which a human could understand the networks process. This improvement dosnt come free however, the cost is a significant accuracy decrcrease when compared to a standard NN. \\

The aim of this project is to explore the idea of LLN's and to investigate wether this accuracy cost can be mitigated but still maintain our ability to understand what the network is using to solve the given problem. This project will evaluate other activation functions based of this idea of Nosiy-OR/AND to see if we can achive a higher accuracy on datasets such as MNIST. Alog with evaluating the accuracy.\\

So far LLN's have only been evaluated on classification datasets (namely MNIST), we will implement and evaluate LLN's to solve a number of other non trivial problems which neural networks are often applyed to.

\section*{3. Proposed Solution}


\section*{4. Evaluating your Solution}


\section*{5. Resource Requirements}


%%%%%%%%%%%%%%%%%%%%%%%%%%%%%%%%%%%%%%%%%%%%%%%%%%%%%%%
\backmatter
%%%%%%%%%%%%%%%%%%%%%%%%%%%%%%%%%%%%%%%%%%%%%%%%%%%%%%%

%\bibliographystyle{ieeetr}
\bibliographystyle{acm}
\bibliography{sample}
\end{document}
